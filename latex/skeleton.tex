% to choose your degree
% please un-comment just one of the following
\documentclass[bsc,frontabs,twoside,singlespacing,parskip,deptreport]{infthesis}     % for BSc, BEng etc.
% \documentclass[minf,frontabs,twoside,singlespacing,parskip,deptreport]{infthesis}  % for MInf
\usepackage{cite}
\begin{document}

\title{Event Ordering In News Articles}

\author{James Robert Friel}

% to choose your course
% please un-comment just one of the following
\course{Artificial Intelligence and Computer Science}
%\course{Artificial Intelligence and Software Engineering}
%\course{Artificial Intelligence and Mathematics}
%\course{Artificial Intelligence and Psychology }   
%\course{Artificial Intelligence with Psychology }   
%\course{Linguistics and Artificial Intelligence}    
%\course{Computer Science}
%\course{Software Engineering}
%\course{Computer Science and Electronics}    
%\course{Electronics and Software Engineering}    
%\course{Computer Science and Management Science}    
%\course{Computer Science and Mathematics}
%\course{Computer Science and Physics}  
%\course{Computer Science and Statistics}    

% to choose your report type
% please un-comment just one of the following
\project{Undergraduate Dissertation} % CS&E, E&SE, AI&L
%\project{Undergraduate Thesis} % AI%Psy
%\project{4th Year Disertation}

\date{\today}

\abstract{}
%TODO


\maketitle

\section*{Acknowledgements}
I'd Like to thank Greggs Bakery for being there for me through tick and thin.

\tableofcontents

%\pagenumbering{arabic}


\chapter{Introduction}
\section{The Problem}
Nominal data is descriptive in nature, making is difficult to assign an cannonical ordering to.
The problem tackled in this disertation is the ordering of news article headlines to generate
a most-probable traversal of a weighted directed graph of these events.
This problem is based off of the paper \cite{abend2015lexical} and some
techniques discussed and used henseforth are based off of this paper.


Our data comes from the Wikipedia ``Today in History'' dataset and the nominal data is retrived
from wikipedia articles.
%The problem explored within this thesis is that of graphing nature.
%Building a most probable path through each node in an edge-weighted directed graph
%, each node beinga single sentence describing an event from history.
%Examples of these events include:
%\say{Julian claendar begins at Greenwich mean noon}
%and \say{AT\&T broken up into 8 countries}

\section{Aims \& Objectives}
The aim of this thesis is to construct a system that predics the most probable path
through an edge-weighted directional graph of events.
we aim to constuct this graph by extracting data from wikipedia for each event and
buld a date estimate from wikipedia. From this data we will conduct several experiments
to maximise the likelyhood of a graph traversal.


The core aims of the project were: 
\begin{itemize}
  \item Generate probable dates For each event
  \item Order these events linearly
\end{itemize}

The further goals of the project were to:
\begin{itemize}
  \item Generate edge-weighted directional graphs of the data
  \item Construct probabilities of maximum spanning walks through the graph
\end{itemize}

Some initial stretch goals of the project were to construct an interactive
interface to the resulting graphs to act as a visual aid for the results.

\section{Testing \& Evaluation}
With our dataset being so large ( around 6250 entries) it can
easily be split into training, devlopment and testing with no
need for overlap. For this the data is split 10\% training,
10\% development and 80\% for testing.

Evaulation of the system was dne using the
Kendall rank correlation coefficient as is a statistic
used to measure the ordinal association between two
measured quantities. This was easy be applied to our results
by compairing the systems estimated date for each event
with the label associated with the event from the data.

\chapter{Background}

\chapter{Related Work \& Motivation}
\section{Motivation}
\section{Related Work}


\chapter{Wikipedia As a Data Source}
While the ``Today in History'' dataset provides an excellent
base for out experiments, the sentences alone
(often between 5 and 15 words) do not contain enough
information to extract relational information from.

In order to build upon this data and procure data that
would aid in our training it was decided to use wikipedia
articles for aditional details.

Wikipedia does contain articles on most any subject and so
finding paragraphs about each event was simple enough. There
were concerns however abgout the legitimacy of the content
available, as any user has the ability to edit an article.
This concern arose from the known conflict-of-interest editing
that occurs on wikipedia.
%Read the wiki paper again anf pad this out
\chapter{Classification \& Regression}

\chapter{Maximum Spanning Graph}

\chapter{Conclusion}

% use the following and \cite{} as above if you use BibTeX
% otherwise generate bibtem entries
\bibliographystyle{plain}
\bibliography{./references.bib}
\end{document}
